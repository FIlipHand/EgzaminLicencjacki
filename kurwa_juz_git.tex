\documentclass[a4paper,12pt,oneside]{book}
\usepackage{polski}
\usepackage[utf8]{inputenc}
\usepackage{graphicx}
\graphicspath{{./images}}
\usepackage[shortlabels]{enumitem}
\usepackage{amssymb}
\usepackage{amsmath}
\usepackage{indentfirst}

\usepackage{tikz}
%\usepackage{etoolbox} % for \ifthen
\usepackage{listofitems} % for \readlist to create arrays
\usetikzlibrary{arrows.meta} % for arrow size
\usepackage[outline]{contour} % glow around text
\contourlength{1.4pt}

\tikzset{>=latex} % for LaTeX arrow head
\usepackage{xcolor}
\colorlet{myred}{red!80!black}
\colorlet{myblue}{blue!80!black}
\colorlet{mygreen}{green!60!black}
\colorlet{myorange}{orange!70!red!60!black}
\colorlet{mydarkred}{red!30!black}
\colorlet{mydarkblue}{blue!40!black}
\colorlet{mydarkgreen}{green!30!black}
\tikzstyle{node}=[thick,circle,draw=myblue,minimum size=22,inner sep=0.5,outer sep=0.6]
\tikzstyle{node in}=[node,green!20!black,draw=mygreen!30!black,fill=mygreen!25]
\tikzstyle{node hidden}=[node,blue!20!black,draw=myblue!30!black,fill=myblue!20]
\tikzstyle{node convol}=[node,orange!20!black,draw=myorange!30!black,fill=myorange!20]
\tikzstyle{node out}=[node,red!20!black,draw=myred!30!black,fill=myred!20]
\tikzstyle{connect}=[thick,mydarkblue] %,line cap=round
\tikzstyle{connect arrow}=[-{Latex[length=4,width=3.5]},thick,mydarkblue,shorten <=0.5,shorten >=1]
\tikzset{ % node styles, numbered for easy mapping with \nstyle
	node 1/.style={node in},
	node 2/.style={node hidden},
	node 3/.style={node out},
}
\def\nstyle{int(\lay<\Nnodlen?min(2,\lay):3)} % map layer number onto 1, 2, or 3

\def\shrug{\texttt{\raisebox{0.75em}{\char`\_}\char`\\\char`\_\kern-0.5ex(\kern-0.25ex\raisebox{0.25ex}{\rotatebox{45}{\raisebox{-.75ex}"\kern-1.5ex\rotatebox{-90})}}\kern-0.5ex)\kern-0.5ex\char`\_/\raisebox{0.75em}{\char`\_}}}

\renewcommand\thesubsection{\arabic{subsection}}
\renewcommand\thechapter{\Roman{chapter}}
\renewcommand\thesection{\arabic{section}}
\renewcommand\thesubsection{(\alph{subsection})}

\begin{document}

	\tableofcontents

	\chapter*{Pytania - dr. hab. Bogdan Księżopolski}
		\addcontentsline{toc}{chapter}{Pytania - dr. hab. Bogdan Księżopolski}
		
		\setcounter{section}{65}
		\section{Charakterystyka kryptografii symetrycznej oraz asymetrycznej.}
		
		
		\setcounter{section}{1}
		\section{Funkcje skrótu (mieszające) i ich zastosowania. }
		
	
	\chapter*{Pytania - dr. hab. Grzegorz Wójcik}
		\addcontentsline{toc}{chapter}{Pytania - dr. hab. Grzegorz Wójcik}
		
		\setcounter{section}{32}
		\section{Budowa sieci neuronowych}
	
	\chapter*{Pytania - reszta}
		\addcontentsline{toc}{chapter}{Pytania - reszta}
		
		\setcounter{section}{0}
		\section{Wektory i macierze – definicje i podstawowe operacje.}
		
		\setcounter{section}{5}
		\section{Sposoby cyfrowej reprezentacji liczby całkowitej i rzeczywistej.}
		
		\setcounter{section}{52}
		\section{Deklaratywne programowanie w logice: klauzule Horne'a, nawracanie.}
	
\end{document}