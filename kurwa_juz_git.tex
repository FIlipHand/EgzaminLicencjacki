\documentclass[a4paper,12pt,oneside]{book}
\usepackage{polski}
\usepackage[utf8]{inputenc}
\usepackage{graphicx}
\graphicspath{{./images}}
\usepackage[shortlabels]{enumitem}
\usepackage{amssymb}
\usepackage{amsmath}
\usepackage{indentfirst}

\usepackage{tikz}
%\usepackage{etoolbox} % for \ifthen
\usepackage{listofitems} % for \readlist to create arrays
\usetikzlibrary{arrows.meta} % for arrow size
\usepackage[outline]{contour} % glow around text
\contourlength{1.4pt}

\tikzset{>=latex} % for LaTeX arrow head
\usepackage{xcolor}
\colorlet{myred}{red!80!black}
\colorlet{myblue}{blue!80!black}
\colorlet{mygreen}{green!60!black}
\colorlet{myorange}{orange!70!red!60!black}
\colorlet{mydarkred}{red!30!black}
\colorlet{mydarkblue}{blue!40!black}
\colorlet{mydarkgreen}{green!30!black}
\tikzstyle{node}=[thick,circle,draw=myblue,minimum size=22,inner sep=0.5,outer sep=0.6]
\tikzstyle{node in}=[node,green!20!black,draw=mygreen!30!black,fill=mygreen!25]
\tikzstyle{node hidden}=[node,blue!20!black,draw=myblue!30!black,fill=myblue!20]
\tikzstyle{node convol}=[node,orange!20!black,draw=myorange!30!black,fill=myorange!20]
\tikzstyle{node out}=[node,red!20!black,draw=myred!30!black,fill=myred!20]
\tikzstyle{connect}=[thick,mydarkblue] %,line cap=round
\tikzstyle{connect arrow}=[-{Latex[length=4,width=3.5]},thick,mydarkblue,shorten <=0.5,shorten >=1]
\tikzset{ % node styles, numbered for easy mapping with \nstyle
	node 1/.style={node in},
	node 2/.style={node hidden},
	node 3/.style={node out},
}
\def\nstyle{int(\lay<\Nnodlen?min(2,\lay):3)} % map layer number onto 1, 2, or 3

\def\shrug{\texttt{\raisebox{0.75em}{\char`\_}\char`\\\char`\_\kern-0.5ex(\kern-0.25ex\raisebox{0.25ex}{\rotatebox{45}{\raisebox{-.75ex}"\kern-1.5ex\rotatebox{-90})}}\kern-0.5ex)\kern-0.5ex\char`\_/\raisebox{0.75em}{\char`\_}}}

\renewcommand\thesubsection{\arabic{subsection}}
\renewcommand\thechapter{\Roman{chapter}}
\renewcommand\thesection{\arabic{section}}
\renewcommand\thesubsection{(\alph{subsection})}

\begin{document}

	\tableofcontents
	\newpage

	\chapter{Pytania - dr. hab. Bogdan Księżopolski}
		
		\setcounter{section}{65}
		\section{Charakterystyka kryptografii symetrycznej oraz asymetrycznej.}
		
		Lorem ipsum dupa dupa
		
		\setcounter{section}{1}
		\section{Funkcje skrótu (mieszające) i ich zastosowania. }
		
		Lorem ipsum dupa dupa
		
		\setcounter{section}{8}
		\section{\color{red} TODO: Protokoły TCP i UDP – porównanie i zastosowanie.}
		
		Lorem ipsum dupa dupa
		
		\setcounter{section}{9}
		\section{\color{red} TODO: Adresowanie w warstwie Internetu modelu TCP/IP.}
		
		Lorem ipsum dupa dupa
		
		\setcounter{section}{11}
		\section{\color{red} TODO: Porównanie modelu OSI i TCP/IP. }
		
		Lorem ipsum dupa dupa
		
		\setcounter{section}{12}
		\section{\color{red} TODO: Mechanizm enkapsulacji w modelu OSI. }
		
		Lorem ipsum dupa dupa
		
		\setcounter{section}{57}
		\section{\color{red} TODO: Mechanizm sesji w zarządzaniu stanem aplikacji sieciowej. }
		
		Lorem ipsum dupa dupa
		
		\setcounter{section}{58}
		\section{\color{red} TODO: Mechanizm gniazd – pojęcie, sposób realizacji i zastosowanie }
		
		Lorem ipsum dupa dupa
		
		\setcounter{section}{59}
		\section{\color{red} TODO: Metody obsługi wielu klientów równolegle w aplikacjach sieciowych.}
		
		Lorem ipsum dupa dupa
		
		\setcounter{section}{60}
		\section{\color{red} TODO: Pocztowe protokoły warstwy aplikacji. }
		
		Lorem ipsum dupa dupa
		
		\setcounter{section}{61}
		\section{\color{red} TODO: Porównanie HTTP i WebSocket. }
		
		Lorem ipsum dupa dupa
		
		\setcounter{section}{62}
		\section{\color{red} TODO: Atrybuty bezpieczeństwa informacji. }
		
		Lorem ipsum dupa dupa
		
		\setcounter{section}{63}
		\section{\color{red} TODO: Modele dystrybucji kluczy kryptograficznych.}
		
		Lorem ipsum dupa dupa
		
		\setcounter{section}{64}
		\section{\color{red} TODO: Rodzaje zagrożeń oraz ochrona aplikacji sieciowych.}
		
		Lorem ipsum dupa dupa
		
		
	
	\chapter{Pytania - dr. hab. Grzegorz Wójcik}
		
		\setcounter{section}{32}
		\section{Budowa sieci neuronowych}
				
				Lorem ipsum dupa dupa
		
		\setcounter{section}{29}
		\section{\color{red} TODO: Modele reprezentacji wiedzy.}
				
				Lorem ipsum dupa dupa
		
		\setcounter{section}{30}
		\section{\color{red} TODO: Mechanizmy wnioskowań. }
				
				Lorem ipsum dupa dupa
		
		\setcounter{section}{31}
		\section{\color{red} TODO: Metody uczenia maszynowego.} 
				
				Lorem ipsum dupa dupa
		
		\setcounter{section}{33}
		\section{\color{red} TODO: Normalizacja baz danych – pierwsza, druga i trzecia postać normalna. }
				
				Lorem ipsum dupa dupa
		
		\setcounter{section}{34}
		\section{\color{green}Paweł \color{red} TODO: Modele baz danych (logiczny, relacyjny, fizyczny). }
				
				Lorem ipsum dupa dupa
		
		\setcounter{section}{35}
		\section{\color{green}Paweł \color{red} TODO: Rodzaje zapytań w języku SQL. }
				
				Lorem ipsum dupa dupa
		
		\setcounter{section}{36}
		\section{\color{green}Paweł \color{red} TODO: Funkcje w języku SQL. }
				
				Lorem ipsum dupa dupa
		
		\setcounter{section}{37}
		\section{\color{green}Paweł \color{red} TODO: Transakcje w bazach danych.} 
				
				Lorem ipsum dupa dupa
		
		
		\setcounter{section}{14}
		\section{\color{red} TODO: Hermetyzacja, dziedziczenie i polimorfizm w programowaniu obiektowym. }
				
				Lorem ipsum dupa dupa
				
		\setcounter{section}{47}
		\section{\color{red} TODO: Główne paradygmaty programowania – charakterystyka i przykłady. }
						
				Lorem ipsum dupa dupa
				
		\setcounter{section}{16}
		\section{\color{red} TODO: Paradygmat i przykłady programowania generycznego (rodzajowego). }
								
						Lorem ipsum dupa dupa
	
	\chapter{Pytania - reszta}
		
		\setcounter{section}{0}
		\section{Wektory i macierze – definicje i podstawowe operacje.}
				
				Lorem ipsum dupa dupa
		
		\setcounter{section}{5}
		\section{Sposoby cyfrowej reprezentacji liczby całkowitej i rzeczywistej.}
				
				Lorem ipsum dupa dupa
		
		\setcounter{section}{52}
		\section{Deklaratywne programowanie w logice: klauzule Horne'a, nawracanie.}
				
				Lorem ipsum dupa dupa
				
		\setcounter{section}{2}
		\section{\color{red} TODO: Problemy rekurencyjne i ich rozwiązywanie. }
		    
		    Lorem ipsum dupa dupa
		
		\setcounter{section}{4}
		\section{\color{red} TODO: Pozycyjne systemy liczbowe i konwersje pomiędzy nimi. }
		    
		    Lorem ipsum dupa dupa
		
		\setcounter{section}{6}
		\section{\color{red} TODO: Typ, zmienna, obiekt i zarządzanie pamięcią. }
		    
		    Lorem ipsum dupa dupa
		
		\setcounter{section}{7}
		\section{\color{green}Paweł \color{red} TODO: Instrukcje sterujące przepływem programu. }
		    
		    Lorem ipsum dupa dupa
		
		\setcounter{section}{10}
		\section{\color{red} TODO: Porównanie zadań przełącznika (switcha) i routera. }
		    
		    Lorem ipsum dupa dupa
		
		\setcounter{section}{13}
		\section{\color{red} TODO: Obiekt i klasa w wybranym języku programowania zorientowanym obiektowo. }
		    
		    Lorem ipsum dupa dupa
		
		\setcounter{section}{15}
		\section{\color{red} TODO: Interfejsy i klasy abstrakcyjne w programowaniu obiektowym. }
		    
		    Lorem ipsum dupa dupa
		
		\setcounter{section}{17}
		\section{\color{red} TODO: Algorytmy sortowania. }
		    
		    Lorem ipsum dupa dupa
		
		
		\setcounter{section}{18}
		\section{\color{red} TODO: Strategia „dziel i zwyciężaj” budowania algorytmów. }
		    
		    Lorem ipsum dupa dupa
		
		\setcounter{section}{19}
		\section{\color{red} TODO: Algorytmy typu zachłannego. }
		    
		    Lorem ipsum dupa dupa
		
		\setcounter{section}{20}
		\section{\color{red} TODO: Algorytmy z nawrotami. }
		    
		    Lorem ipsum dupa dupa
		
		\setcounter{section}{21}
		\section{\color{red} TODO: Grafy, drzewa, kopce – charakterystyka i przykłady zastosowania. }
		    
		    Lorem ipsum dupa dupa
		
		\setcounter{section}{46}
		\section{\color{green}Paweł \color{red} TODO: Definicja i klasy złożoności obliczeniowej – czasowej i pamięciowej. }
		    
		    Lorem ipsum dupa dupa
		
		\setcounter{section}{55}
		\section{\color{red} TODO: Kodowanie liczb ze znakiem w systemie U2, generowanie liczby ze znakiem przeciwnym, dodawanie i odejmowanie. }
		    
		    Lorem ipsum dupa dupa
		
	\chapter{Pytania których raczej nie dostaniemy}
	
	\setcounter{section}{27}
	\section{28. Różnice pomiędzy obsługą zdarzeń w przerwaniach sprzętowych a obsługą zdarzeń w pętli programowej.}
					
					Lorem ipsum dupa dupa
	
	\setcounter{section}{28}
	\section{\color{red} TODO: Powody i przykłady stosowania mikrokontrolerów zamiast typowych komputerów. }
					
					Lorem ipsum dupa dupa
	
	\setcounter{section}{38}
	\section{\color{red} TODO: Standardowe metodyki procesu wytwórczego oprogramowania. }
					
					Lorem ipsum dupa dupa
	
	\setcounter{section}{39}
	\section{\color{red} TODO: Metodyki zwinne – SCRUM. }
					
					Lorem ipsum dupa dupa
	
	\setcounter{section}{40}
	\section{\color{red} TODO: Testowanie oprogramowania. }
					
					Lorem ipsum dupa dupa
	
	\setcounter{section}{41}
	\section{\color{red} TODO: Diagramy UML. }
					
					Lorem ipsum dupa dupa
	
	\setcounter{section}{42}
	\section{\color{red} TODO:  Wzorce projektowe programowania obiektowego. }
					
					Lorem ipsum dupa dupa
	
	\setcounter{section}{43}
	\section{\color{red} TODO: Definicja funkcji obliczalnej (częściowo rekurencyjnej). }
					
					Lorem ipsum dupa dupa
	
	\setcounter{section}{44}
	\section{\color{red} TODO: Maszyna Turinga jako model procesów obliczalnych. }
					
					Lorem ipsum dupa dupa
	
	\setcounter{section}{45}
	\section{\color{red} TODO: Zagadnienia nierostrzygalne w kontekście obliczalności. }
					
					Lorem ipsum dupa dupa
	
	\setcounter{section}{48}
	\section{\color{red} TODO:  Gramatyki bezkontekstowe – definicje, charakterystyki i przykłady. }
					
					Lorem ipsum dupa dupa
	
	\setcounter{section}{49}
	\section{\color{red} TODO: Analiza leksykalna, syntaktyczna i semantyczna kodu. }
					
					Lorem ipsum dupa dupa
	
	\setcounter{section}{50}
	\section{\color{red} TODO: Rodzaje błędów w kontekście analizy leksykalnej, syntaktycznej i semantycznej kodu.}
					
					Lorem ipsum dupa dupa
					
	\setcounter{section}{51}
	\section{\color{red} TODO: Deklaratywne programowanie funkcyjne: rachunek lambda, monady. }
					
					Lorem ipsum dupa dupa
	
	\setcounter{section}{53}
	\section{\color{red} TODO: Podstawowe układy systemu mikroprocesorowego i sposób wymiany informacji pomiędzy nimi. }
					
					Lorem ipsum dupa dupa
	
	\setcounter{section}{54}
	\section{\color{red} TODO: Dekoder, multiplekser i demultiplekser: budowa, zasada, działania, przeznaczenie, zastosowanie. }
					
					Lorem ipsum dupa dupa
	
	\setcounter{section}{56}
	\section{\color{red} TODO: Budowa i zasada działania generatora obrazu w systemie mikroprocesorowym. }
					
					Lorem ipsum dupa dupa
	
	\setcounter{section}{56}
	\section{\color{red} TODO: Wielowarstwowa organizacja systemów komputerowych. }
						
						Lorem ipsum dupa dupa
	
	\setcounter{section}{56}
	\section{\color{red} TODO: System operacyjny – charakterystyka, zadania, klasyfikacja. }
						
						Lorem ipsum dupa dupa
	
	\setcounter{section}{56}
	\section{\color{red} TODO: Procesy i wątki – charakterystyka i problemy. }
						
						Lorem ipsum dupa dupa
	
	\setcounter{section}{56}
	\section{\color{red} TODO: Zarządzanie pamięcią operacyjną w systemie operacyjnym. }
						
						Lorem ipsum dupa dupa
	
	\setcounter{section}{56}
	\section{\color{red} TODO: Organizacja systemu plików i pamięci zewnętrznej. }
						
						Lorem ipsum dupa dupa
						
	\setcounter{section}{56}
	\section{\color{red} TODO: Podstawowe charakterystyki statystyki opisowej i matematycznej.}
							
							Lorem ipsum dupa dupa
	
	
	
	
\end{document}